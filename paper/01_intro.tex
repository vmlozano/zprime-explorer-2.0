Z' models are widely used in a variety of context, as they are theoretically motivated (portal model $F_{\mu \nu} F^{\prime, \mu \nu}$). Come from string theory. Model used for flavour physics, for low resonance searches (dark photon), can explain LFU anomalies ($U(1)_{B-L}$, for instance, or $U(1)_{\tau \mu}$) and also used as mediators to the dark sector [cite s-channel review]. 

In particular, these models are also of interest in the context of dark matter, due to the process $ p p \to Z' \to \chi \chi$ via a Z' in the s-channel. The presence of this process guarantees that Z' couples to both the Standard Model and the dark sector, thus given a robust foundation to a mediator search program at the LHC, based in both visible and "invisible" final states. [cite s-channel models, cite some ATLAS and CMS papers]

Given the ubiquitious nature of Z' models in a plethora of BSM context, it is relevant to have at hand a fast and flexible tool for the fast reinterpretation of the broad palette of experimental searches. A first step in that direction was done in Z' 1.0, where the whole suite of visible final states was introduce. In comparison with its main rivals, like ZPEED (cu�l otro?) or recasting software like CheckMATE (where events are needed), the philosophy of Z' explorer is to \JZ{here for Rosa}. Based on a simple, yet generic parametrization of Z' couplings and widths, it allows the user to go beyond the default benchmark models and explorer a richer landscape of Z' models (here mention that ATLAS uses only vector or axial-vector), allowing to test arbitrary coupling structures. [discuss more details of Z' 1.0, and the competitors]

In this article we extend the capabilities of Z' explorer to further include searches with missing energy. To that effect, we perform the first reintepretation of the ATLAS missing-energy + jets study [CITE], which would give the strongest constraints in the invisible channels via the mono-jet process, $p p \to Z' j, Z' \to \chi \chi$. Our code goes beyond the two studied ATLAS scenarios, including the possibility of an arbitrary coupling structure for the dark sector to the Z' boson, going beyond the axial-vector and axial scenarios used in the ATLAS study. Our results are implemented in the Z' 2.0, which is publicly available in [CITE link]. As a by-product of our analysis, we have derived for the first time (to the best of our knowledge) the complete analytical expression for exclusion in the case of unknown background, irrespectively of being in a Poissonian or Gaussian case;

The current article is structure as follows. In section 2 we review the fundamentals of Z' models, presenting the parametrization used within Z' explorer, which is in correspondence with the Z' s-channel





